\chapter{Introduzione}

I numeri casuali hanno un ruolo importante in diversi campi, come la crittografia, le simulazioni scientifiche o le lotterie. Ogni giorno, spesso inconsciamente, utilizziamo gli output di generatori di numeri casuali. Basti pensare alle One-time password (OTP), ai pin inseriti per operazioni bancarie, o ai Captcha che spesso ci viene richiesto di identificare per poter accedere ad un sito.
Il loro utilizzo si basa sull'impredicibilità del loro valore, caratteristica che spesso non può essere garantita con i generatori attualmente utilizzati. In informatica, i generatori di numeri casuali (RNGs, Random Number Generators) sono basati su algoritmi di generazione pseudo-casuali, che creano il numero "espandendo" un seme in modo deterministico. Nonostante infatti la sequenza (binaria) in uscita da questi generatori sia costituita in modo bilanciato da 0 e 1, una correlazione tra il numero generato e il seme, seppur di difficile identificazione, esiste. Questo determinismo va ovviamente a diminuire la sicurezza della crittografia tradizionale, in quanto un attaccante, conoscendo sia l'algoritmo utilizzato che il seme e con a disposizione sufficienti risorse computazionali, sarebbe in grado di predire i numeri pseudo-casuali generati. 

La soluzione per ottenere dei veri numeri casuali sarebbe quella di sfruttare la casualità associata a particolari fenomeni naturali come rumore atmosferico, rumore termico, sistemi caotici o rumore in circuiti elettrici. Tali fenomeni sono talmente complicati che possono essere sfruttati per generare, almeno apparentemente, numeri casuali non deterministici. Tuttavia, la qualità di tali valori è di difficile quantificazione e non risulta possibile scartare l'ipotesi che, in un futuro non troppo lontano, sia sviluppata una teoria che modelli precisamente questi fenomeni naturali, rendendo i numeri generati predicibili.

L'unico modo per essere certi della casualità dei valori generati è basarsi su fonti con casualità intrinseca ad esse associata. Come verrà descritto in questa trattazione, la meccanica quantistica è non deterministica di natura, quindi un sistema quantico risulta essere una fonte ideale per generare numeri realmente casuali. Inoltre, le leggi della meccanica quantistica possono essere utilizzate per quantificare la qualità dei valori generati. In questo documento verranno presentate le differenti categorie di generatori quantici di numeri casuali (Quantum Random Number Generators) e verranno esaminate possibili applicazioni degli stessi nel campo della crittografia, mostrando come il loro utilizzo potrebbe rinforzare la sicurezza di protocolli classici.