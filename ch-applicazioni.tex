\chapter{Applicazioni dei QRNG nei protocolli crittografici tradizionali}

Gli algoritmi di cifratura vengono utilizzati su ampia scala, dunque è importante capire cosa li rende sicuri e cosa si potrebbe fare per incrementarne l'affidabilità. Tutti i protocolli crittografici tradizionali sono deterministici e di conseguenza reversibili, pertanto l'unica vera fonte di sicurezza è rappresentata da una parte non deterministica: una chiave o un valore a singolo utilizzo che dovrebbe essere casuale. La qualità e la dimostrabilità della casualità di questo insieme di bit diventano dunque cruciali per la sicurezza dell'intero sistema. 

\section{Sistemi di crittografia}
Un algoritmo crittografico ha lo scopo di trasformare testo in testo cifrato per successivamente riconvertire il testo cifrato in quello originale. Questa procedura stabilisce un canale di comunicazione sicuro tra due entità in un dominio pubblico, come ad esempio Internet, dove sono presenti anche utenti non autorizzati o malintenzionati. 

Ogni sistema crittografico include tre processi fondamentali: il Key Schedule Algorithm (KSA), l'Encryption Algorithm (EA) e il Decryption Algorithm (DA). Il KSA è un processo di generazione chiavi per la crittografia e decrittografia. Richiede una chiave di partenza come insieme di bit, ad esempio 40, 128, 192, 256, o anche un numero più significativo, per poi espanderli in base alle fasi di elaborazione o al numero di round progettati per l'algoritmo. Lo scopo del KSA è rendere la chiave abbastanza forte da non essere vulnerabile ad attacchi e fare in modo che nessuno possa risalire all'input di partenza. L'EA, algoritmo di cifratura, converte i dati in un formato illeggibile utilizzando la chiave prodotta dal KSA. Dall'altro lato, un algoritmo di decifratura (DA) prevede l'utilizzo della stessa o di un altra chiave per la decodifica, ovvero la riconversione del testo cifrato nei dati originali. 

La sicurezza di un cifratore dipende da quanto la chiave utilizzata è vulnerabile ad attacchi di crittoanalisi. In aggiunta al KSA, che divide le chiavi in sottoparti, possono essere utilizzati numeri random per rendere le chiavi più robuste e complesse.
\begin{figure}[h]
    \centering
    \includegraphics[scale=0.5]{immagini/ksa con random number.png}
    \caption{Key Schedule Algorithm (KSA) e fase di cifratura di un sistema crittografico basato su RNG \cite{saini_quantum_2022}. }
    \label{fig: KSA con RNG}
\end{figure}
Come mostrato in \ref{fig: KSA con RNG}, i Random Number Generators (RNGs) forniscono i numeri random utilizzati per ottenere causalità nei sistemi di crittografia. L'incorporazione di RNG nei KSA e durante la fase di cifratura migliora diversi parametri nei crittosistemi, che variano in base al tipo di generatore utilizzato, come vedremo nella sezione \ref{subsec-crittosistemiRNG}. 

La differenza principale che sussiste tra i diversi algoritmi crittografici consiste nella relazione tra la chiave di cifratura e la chiave di decifratura. Gli algoritmi classificati come simmetrici utilizzano la stessa chiave privata per entrambe le fasi. Invece, la crittografia asimmetrica si avvale di una coppia di chiavi, una privata ed una pubblica, utilizzate per cifratura e decifratura. 

\begin{figure}[h]
    \centering
    \includegraphics[scale=0.4]{immagini/crittografia simmetrica e asimmetrica.png}
    \caption{(a) Crittografia simmetrica; (b) Crittografia asimmetrica \cite{saini_quantum_2022}.}
\end{figure}

Un ultimo elemento necessario all'utilizzo degli algoritmi di cifratura, sono i metodi per lo scambio delle chiavi che definiscono dei protocolli sicuri per permettere a due entità di scambiarsi una chiave in totale segretezza. Tali protocolli consistono sostanzialmente nello scambio di una serie di messaggi tra due entità al fine di concordare i valori con cui costruire la chiave. Tali messaggi sono appositamente studiati per fare in modo che chiunque ascolti il canale non abbia modo di riprodurre la chiave finale generata dalle due entità. Esistono diversi metodi per la generazione di chiavi condivise, due tra i più famosi sono RSA e Diffie-Hellman che verranno illustrati più nel dettaglio nella sezione \ref{subsec:scambio-chiavi}.

\section{Crittosistemi basati su RNGs}
\label{subsec-crittosistemiRNG}
In questo paragrafo verranno presentati diversi algoritmi di cifratura e di scambio chiavi nei quali sono stati integrati generatori di numeri random di vario genere. Lo scopo di questa sezione è quello di sottolineare i miglioramenti apportati grazie all'utilizzo di RNG all'interno di protocolli di sicurezza tradizionali. In ciascuno dei sistemi che verranno citati è possibile implementare generatori quantici di numeri random al fine di ottenere risultati ancora più vantaggiosi. 

\subsection{Algoritmi di cifratura}

\subsubsection{BloStream} 
Gli stream ciphers sono cifratori a chiave simmetrica ampiamente utilizzati per la cifratura di dati sensibili ad alta velocità, ad esempio quando i dati devono essere trasmessi tramite un canale di comunicazione o tramite un browser web. La struttura di base di uno stream cipher richiede la generazione di una keystream, ovvero una sequenza di cifre pseudocasuali. Per la generazione del testo cifrato, ogni cifra in chiaro viene combinata con una cifra della keystream, generalmente utilizzando "Exclusive or" (XOR). La stessa keystream viene poi posta in XOR con il testo cifrato per poter recuperare l'originale.   Gli stream cipher sono stati progettati per avere determinate caratteristiche di sicurezza ed efficienza, tuttavia la loro velocità non è sufficientemente alta per poter evitare attacchi di sicurezza informatica. Apprensioni di questo tipo hanno spinto la comunità informatica a prediligere l'utilizzo di block cipher. 

BloStream \cite{kashmar_blostream_2017} è uno stream cipher ad alta velocità disegnato per essere più flessibile, veloce, casuale e sicuro rispetto agli stream cipher convenzionali che utilizzano lo XOR nella fase di combinazione. Tali risultati sono stati ottenuti grazie all'utilizzo di un PRNG ad alte prestazioni. Con un generatore di numeri pseudo casuali, utilizzando un algoritmo Rabbit mescolato con un combinatore con funzione di arrotondamento invertibile non lineare, diventa estremamente complesso calcolare la keystream o recuperare il testo in chiaro. Uno studio approfondito \cite{kashmar_blostream_2017} ha dimostrato come l'introduzione di un PRNG in BloStream abbia perfezionato la velocità e i requisiti di memoria dell'algoritmo rispetto ad una serie di altri cifrari attualmente in uso tra cui RC4, Chameleon e Serpent. Inoltre, si è dimostrato considerevolmente resistente a numerosi attacchi informatici tra cui attacchi di forza bruta, attacchi statistici, attacchi differenziali, che sono stati applicati al cifratore.  

\subsubsection{Present}
La Lightweight Cryptography (LWC) gioca un ruolo fondamentale nell'ottenimento di elevati livelli di sicurezza quando si hanno a disposizione bassi livelli di energia, memoria e capacità computazionale. Nell'ambito della crittografia simmetrica, troviamo adatti a questo scopo i block cipher. Un block cipher processa l'input suddividendolo in blocchi costituiti da un numero prefissato di bit; a differenza degli stream cipher che cifrano un elemento alla volta, questo algoritmo cifra gli elementi presenti in un blocco contemporaneamente. 

L'Advanced Encryption Standard (AES) è il block cipher più diffuso e rappresenta una parte fondamentale di numerosi sistemi di sicurezza. Tuttavia, AES è utilizzato nei processori ad alte prestazioni e non risulta adatto alle piattaforme che presentano vincoli in termini di risorse. Di conseguenza, sono stati sviluppati block cipher più leggeri al fine di ottenere il miglior compromesso tra sicurezza e potenza. Tra questi troviamo PRESENT, un cifrario a blocchi che spicca tra gli altri grazie alla sua efficienza hardware ed è anche standardizzato da ISO/IEC 29192-2. L'algoritmo si avvale dell'utilizzo di un modulo per la generazione di chiavi casuali da 80 bit che combina due generatori di numeri random: TRNG e PRNG. Tale modulo rende il valore chiave utilizzato nell'algoritmo impredicibile da parte di attaccanti malintenzionati, inoltre l'architettura PRESENT-TRNG-PRNG si è dimostrata \cite{kowsalya_low_2021} più performante rispetto ad altre configurazioni possibili dell'algoritmo. Dunque, l'implementazione di generatori di numeri random all'interno del cifratore PRESENT ha portato ad un miglioramento sia in termini di prestazioni che di sicurezza.  

\subsubsection{Chaos-based AES (CCAES)}
Con l'avanzamento delle tecnologie di comunicazione, la trasmissione di immagini digitali è diventata un problema comune in quanto esse rappresentano il 70\% dei dati scambiati via Internet. Gli algoritmi di cifratura svolgono un ruolo essenziale per assicurare uno scambio sicuro delle immagini, tuttavia per essere considerati affidabili devono non solo offrire elevati livelli di sicurezza ma anche un basso tempo di esecuzione. Per questi motivi, la maggior parte degli algoritmi di cifratura di immagini presenti sul mercato non sono adatti ad applicazioni nelle quali la sicurezza delle comunicazioni è cruciale. 

CCAES \cite{arab_image_2019} è un algoritmo di cifratura delle immagini che combina la tecnica di chaos sequence per la generazione di numeri casuali con l'algoritmo AES modificato. In questo metodo la chiave di cifratura viene generata dalla sequenza del caos di Arnold. Quindi, l'immagine di partenza viene crittografata utilizzando l'algoritmo AES modificato ed implementando le round key prodotte dal sistema caos. La teoria del caos è una branca della matematica che studia i sistemi estremamente complicati. In questi sistemi, l'applicazione di piccoli cambiamenti (apparentemente ignorabili) nell'input determina cambiamenti significativi nell'output. Se una persona non autorizzata non conosce i parametri di controllo e i valori iniziali corretti, non può indovinare la sequenza del caos. 
L'algoritmo AES è un block cipher simmetrico ed esiste in tre varianti: AES-128, AES-192, AES-256. Ogni variante, cifra e decifra i dati in blocchi da 128 bit utilizzando chiavi crittografiche rispettivamente di 128, 192 e 256 bit. L'algoritmo è basato sull'esecuzione di diversi round, ognuno dei quali consiste in diverse fasi di elaborazione che includono la sostituzione, la trasposizione e la combinazione del testo in chiaro al fine di trasformarlo in testo cifrato. L'AES modificato sviluppato per CCAES consiste in 10 round di cifratura e le operazioni di sostituzione ed integrazione delle colonne sono state rimpiazzate dalla conversione lineare e dalla somma dei valori dei pixel. Queste operazioni non solo riducono la complessità temporale dell'algoritmo ma aggiungono anche capacità di diffusione all'algoritmo CCAES, rendendo le imamgini crittografate più difficilmente predicibili. Tali fattori, ai quali va aggiunta la grande sensibilità ai valori di input di questo approccio, consentono all'algoritmo di resistere agli attacchi differenziali. Inoltre, il key space previsto dall'algoritmo è abbastanza grande da resistere ad attacchi di forza bruta. In conclusione, l'algoritmo CCAES si è dimostrato \cite{arab_image_2019} molto più veloce rispetto all'algoritmo AES tradizionale e più sicuro e resistente agli attacchi informatici rispetto agli algoritmi di cifratura immagini comunemente utilizzati. 

\subsection{Algoritmi di scambio chiavi}
\label{subsec:scambio-chiavi}

\subsubsection{Chaos-based hybrid RSA (CRSA)}
RSA è un algoritmo di crittografia asimmetrico comunemente utilizzato per cifratura e scambio di chiavi. L'algoritmo prevede quattro step: la generazione delle chiavi, lo scambio delle stesse, una fase di cifratura e la conseguente decifratura. Un principio alla base di RSA è l'osservazione che è pratico trovare tre interi positivi molto grandi $e$, $d$ e $n$, tali che con esponenziazione modulare per tutti gli interi m (con $0 \le m < n$):
\begin{equation}
    (m^{e})^{d}\equiv m{\pmod {n}}
\end{equation}
e che conoscendo $e$ e $n$, o anche $m$, può essere estremamente difficile trovare $d$. Inoltre, per alcune operazioni è conveniente che si possa cambiare l'ordine delle due esponenziazioni e che questa relazione implichi anche:
\begin{equation}
    (m^{d})^{e}\equiv m{\pmod {n}}.
\end{equation}
RSA prevede una chiave pubblica e una chiave privata, la chiave pubblica può essere conosciuta da tutti e viene utilizzata per crittografare i messaggi. L'intenzione è che i messaggi crittografati con la chiave pubblica possano essere decifrati solo in un ragionevole lasso di tempo utilizzando la chiave privata. La chiave pubblica è rappresentata dagli interi $n$ ed $e$, e la chiave privata dall'intero $d$. 

Uno dei metodi più significativi utilizzati negli studi per il mantenimento della sicurezza dei dati è l'uso di sistemi caotici nella crittografia, come appena visto anche nel caso di CCAES. Infatti, a causa della sensibile dipendenza dalle condizioni iniziali e dai parametri di controllo dei sistemi caotici, è stato affermato che la scienza del caos e la crittografia sono strettamente accoppiate. CRSA \cite{cavusoglu_design_2017} è un algoritmo di crittografia ibrido che utilizza algoritmi RNG e RSA chaos-based. Per lo sviluppo di questo algoritmo è stato sviluppato un nuovo sistema caotico con una complessa struttura dinamica, al fine di evitare problemi di sicurezza legati all'utilizzo di sistemi con una struttura dinamica già ben conosciuta da tutti. Su tale sistema è stato disegnato il generatore di numeri random, sottoposto a test NIST e FIPS \cite{cavusoglu_design_2017} che rappresentano gli standard più elevati per l'affidabilità dei numeri casuali generati. L'algoritmo CRSA nel suo complesso è stato utilizzato in applicazioni crittografiche ed i risultati raccolti sono stati confrontati con quelli prodotti dall'algoritmo RSA tradizionali privo di RNG chaos-based. L'analisi \cite{cavusoglu_design_2017} dimostra che CRSA dispone di un più elevato livello di sicurezza, in particolare sono stati migliorati i seguenti parametri: istogramma, correlazione, sensibilità delle chiavi, key space ed entropia delle informazioni. 

\subsubsection{Diffie-Hellman con QRNG}
Il protocollo di scambio chiavi Diffie-Hellman offre una soluzione al seguente problema: due entità vogliono condividere una chiave segreta, ma il loro canale di comunicazione non è sicuro e dunque tutte le informazioni potrebbero essere intercettate da un utente malintenzionato. Il primo passo dell'algoritmo prevede che le due entità, Alice e Bob, si accordino per un numero intero $p$ arbitrariamente grande, ed un intero $root = g \: modulo \: p$. I due numeri sono resi pubblici. Nel secondo passo, Alice sceglie un intero $a$ che mantiene segreto e Bob fa la medesima cosa selezionando un intero $b$. Alice e Bob usano il numero segreto per calcolare: 
\begin{equation}
    A \equiv g^{a} \: \mathrm{mod} \: p, \; B \equiv g^b \: \mathrm{mod} \: p.
\end{equation}
Nell'ultimo step, le due parti si scambiano i valori calcolati ed alla fine saranno entrambi in possesso della medesima chiave di cifratura, ottenuta come di seguito: 
\begin{equation}
    A : B^a \: \mathrm{mod} \: p, \; B: A^b \: \mathrm{mod} \: p.
\end{equation}

Nell'esperimento riportato dal documento \cite{mogos_use_2016}, è stato utilizzano un Quantum Random Number Generator (QRNG) per il calcolo dei numeri segreti $a$ e $b$. Il QRNG utilizzato è Quantis, prodotto da IdQuantique Company, Svizzera. Quantis è un generatore di numeri casuali quantistici che sfrutta un processo quantistico ottico come fonte di casualità, secondo il processo illustrato nella sezione \ref{QRNG_ottici}. 
I generatori di numeri casuali quantistici hanno il vantaggio rispetto alle fonti di casualità convenzionali di essere invulnerabili alle perturbazioni ambientali e di consentire la verifica dello stato in tempo reale. Inoltre, i QRNG, come sottolineato più volte in questa trattazione, sono fonti di numeri casuali reali utilizzabili in applicazioni con gli standard di sicurezza più rigorosi. Questi fattori rendono Diffie-Hellman potenziato con un QRNG un sistema crittografico praticamente non vulnerabile.  

\section{Analisi costi/benefici dell'utilizzo di QRNG nei sistemi crittografici}
\subsection{Vantaggi}
Come appena illustrato, i numeri casuali vengono utilizzati nei sistemi crittografici come semi per la generazione di chiavi, sia nel caso degli algoritmi di cifratura sia per quelli di scambio chiavi. L'affidabilità delle chiavi generate dipende dalla casualità del seme di input, portando l'intera sicurezza del sistema a dipendere anche dai RNG. Nella sezione precedente abbiamo sottolineato i vantaggi apportati dall'utilizzo di un qualsiasi tipo di generatore di numeri random all'interno dei crittosistemi classici. In Tabella \ref{tab:comparazione-rng} presenteremo invece i vantaggi dell'utilizzo di un QRNG rispetto ai RNG tradizionali. 

Ciò che separa i QRNG dai generatori classici di numeri random è la capacità di generare numeri puramente casuali. I PRNG e i TRNG classici risultano vulnerabili agli attacchi informatici per la loro predicibilità. Infatti, i numeri casuali prodotti dai tradizionali RNG si basano su un input che risulta essere predicibile, così come l'output generato. Di conseguenza, tutti i numeri generati possono essere riprodotti successivamente anche grazie a tecniche innovative di machine learning ed attacchi operati da computer quantici. Questa problematica viene superata dai QRNG che sono basati su fonti intrinsecamente random. \\
I vantaggi dell'integrazione di un QRNG in un sistema crittografico possono essere riassunti dai seguenti punti: 
\begin{itemize}
    \item si affidano alle caratteristiche quantistiche per generare una casualità genuina;
    \item la fonte di casualità è imprevedibile e controllata dal processo quantistico. Ciò è in netto contrasto con i tradizionali RNG che si basano su sconosciuti, ma, in principio, conoscibili dati impliciti preesistenti in un dispositivo fisico, informazioni che potrebbero anche essere parzialmente impiantate;
    \item la certificazione e la convalida sono aiutate dai processi fisici di natura relativamente semplice alla base dei QRNG;
    \item è possibile ed estremamente efficace il monitoraggio in tempo reale della sorgente di entropia;
    \item tutti gli attacchi alla sorgente di entropia sono rilevabili;
    \item sono inespugnabili da computer quantistici.
\end{itemize}
Tutti i punti appena elencati sottolineano che i generatori quantici di numeri random sono essenzialmente sicuri. 

\begin{table}
    \centering
    {\renewcommand{\arraystretch}{2}%
    \begin{tabularx}{\textwidth}{| l | X | X |}
    \hline
    \textbf{Proprietà} & \textbf{Tradizionali} & \textbf{Quantum}  \\ \hline
    Sorgente di entropia  &  Casualità basata sulla complessità del processo e ignoranza parziale.  & Casualità genuina. \\ \hline
    Facilità di certificazione  &  Capacità limitata di certificare il processo fisico sottostante, essendo questo intrinsecamente complesso. Certificazione della qualità della produzione basata su prove standard.  &  Può convalidare i processi fisici sottostanti. Certificazione della qualità della produzione basata su prove standard. \\ \hline
    Resistenza alla manomissione  &  Una certa capacità di eseguire un controllo sullo stato della fonte di entropia.  &  Controllo integrato basato sulla semplicità del processo e più sensibile alla manomissione. Le versioni Device-Independent offrono la massima resistenza contro la manomissione della sorgente di entropia, anche da parte dei fornitori stessi. \\ \hline
    Qualità dell'entropia  &  Vari gradi. Il processo sottostante utilizzato come fonte di entropia può funzionare in un regime fisico in cui vi sono grandi bias e correlazioni relativamente elevate, ovvero bassa entropia. &  Elevata entropia sin dall'inizio basata sulla natura quantica della sorgente. \\ \hline
    Velocità  &  Può essere molto alta e possono essere combinate diverse fonti per ottenere tassi ancora più elevati. &  Alta, anche a causa della qualità dell'entropia iniziale, ma le implementazioni Device-Independent possono essere lente. \\ \hline
    Dimensioni  &  Può essere molto piccolo e integrato in un chip, ad esempio sfruttando una fonte di casualità come il rumore termico.  &  Varia in modo sostanziale, passando da incorporabile negli smartphone a dimensioni comparabili con una stanza nel caso di QRNG Device-Independent basati sulla non località. \\ \hline
    \end{tabularx}}
    \caption{Confronto tra generatori di numeri casuali tradizionali e quantici \cite{piani_quantum_nodate}}
    \label{tab:comparazione-rng}
\end{table}

\subsection{Costi}
Attualmente, il costo delle tecnologie QRNG è relativamente alto rispetto ad alternative come i tradizionali TRNG e PRNG, per i quali le implementazioni sono spesso incluse gratuitamente nei processori o come parte dei sistemi operativi. Si crea dunque una barriera alla creazione di un mercato di QRNG a causa della pletora di offerte alternative a basso costo che forniscono numeri casuali con vari gradi di sicurezza. I QRNG si presentano come la scelta di sicurezza più elevata; tuttavia, agli occhi dei designer di prodotti che cercano di soddisfare un particolare standard di sicurezza, se l'utilizzo di un PRNG consente loro di accedere al rispettivo mercato non risulta conveniente spendere denaro aggiuntivo per soddisfare il livello di sicurezza più elevato. 

Un altro fattore di costo da tenere in considerazione prima di utilizzare generatori quantici di numeri casuali riguarda la velocità e le dimensioni. Sino ad ora abbiamo parlato della proprietà unica dei QRNG di certificare che i numeri generati siano validi e privati. Tuttavia, questa peculiarità è propria solo dei generatori quantici self-tested. Questo tipo di dispositivi, sebbene dispongano delle caratteristiche che si cercano in un generatore quantico, presentano anche degli svantaggi. Infatti, come discusso nella sezione \ref{self_testing_qrng}, i self-tested QRNG possono essere molto ingombranti e raggiungere velocità non sufficienti per soddisfare i criteri di molti sistemi. Dall'altro lato, le altre tipologie di QRNG riescono a sorpassare tali problematiche, ma possiedono una limitazione fondamentale: è impossibile per l'utente verificare che i numeri generati siano genuinamente random. 

Infine, un inibitore all'integrazione di QRNG nei sistemi crittografici riguarda le certificazioni. Per le certificazioni dei prodotti crittografici ai governi statunitense e canadese viene utilizzato lo standard di valutazione della sicurezza FIPS140. Tale standard è anche ufficiosamente utilizzato dal settore finanziario del Nord America per prendere decisioni in merito all'acquisto di moduli di sicurezza hardware (HSM). FIPS140 stabilisce un livello di sicurezza minimo per le implementazioni crittografiche e si concentra sul test dei prodotti rispetto a un set standard di input e sulla valutazione degli output e del comportamento del prodotto. I fornitori tendono a concentrarsi solo sull'insieme minimo definito di funzionalità di sicurezza per ottenere il livello di certificazione che stanno perseguendo, poiché l'aggiunta di funzionalità aggiuntive oltre a ciò comporta un rischio negativo che il prodotto possa non superare la valutazione. A meno che i QRNG non siano specificamente aggiunti e richiesti dallo standard FIPS140, i fornitori tenderanno a scegliere il percorso meno costoso per la certificazione. Ciò potrebbe inibire l'adozione di QRNG per i settori governativi nel mercato nordamericano.


\subsection{Considerazioni finali}
Tenendo in considerazione i vantaggi ed i costi appena discussi, ne risulta che i QRNG potrebbero essere la soluzione ideale in molte circostanze grazie alle loro proprietà fondamentali, ma non in tutte. L'integrazione di QRNG è particolarmente adatta nei casi in cui i guadagni ottenuti in termini di sicurezza sono sufficienti a giustificare i maggiori costi e la velocità di generazione inferiore che li caratterizza. In particolare, l'integrazione di generatori quantici di numeri casuali nei sistemi crittografici tradizionali risulta vantaggiosa nel caso di applicazioni che proteggono asset di alto valore e sistemi critici: queste potrebbero includere informazioni riservate (mediche e finanziarie) da condividere in forma crittografata o autenticazione per l'accesso ad applicazioni e database strategici presso le forze armate, a livello governativo o aziendale. La casualità fornita dai QRNG Device-Independent può assumere un'importanza particolare per tali risorse o sistemi. Dall'altro lato, le proprietà uniche dei QRNG non sono sempre richieste. Per le applicazioni in cui la segretezza non è un problema, come l'esecuzione di simulazioni, il premio pagato per i QRNG potrebbe non essere un valore giustificabile. Inoltre, nel caso in cui i protocolli di sicurezza prevedano un algoritmo di scambio chiavi nel quale entrambe le parti giocano un ruolo fondamentale nella creazione della chiave condivisa, per assicurare i benefici dell'utilizzo della tecnologia quantica per la generazione di numeri casuali è fondamentale che entrambe le entità in questione dispongano di un QRNG. Ad esempio, nel caso dell'algoritmo di key exchange Diffie-Hellman, se un attaccante dispone di un certo numero di informazioni e riesce a risalire ad uno dei due numeri casuali generati dalle rispettive entità, questo gli basterà per compromettere il sistema, anche se l'altro numero rimane sconosciuto.




