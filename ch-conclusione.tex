\chapter{Conclusione}
In questa trattazione abbiamo presentato una panoramica sulle diverse tipologie di RNG ed il loro ruolo nei sistemi crittografici. Inoltre, sono state descritte le diverse tipologie di QRNG ed è stata effettuata un'analisi sui vantaggi che questi potrebbero apportare se integrati negli algoritmi di cifratura e scambio chiavi attualmente in uso. 

Lo studio ha dimostrato che i generatori di numeri random forniscono numeri casuali fondamentali per la sicurezza dei crittosistemi e ne migliorano la sicurezza rendendo i diversi cifratori robusti e resistenti ai vari attacchi informatici. Inoltre, i Quantum Random Number Generator (QRNG) forniscono una casualità genuina, al contrario dei PRNG e TRNG classici. Esistono diverse categorie di generatori quantici, distinte in base alla dipendenza o meno del dispositivo da specifiche implementazioni fisiche. Tali categorie sono: trusted device QRNG, self-testing QRNG e semi self-testing QRNG.

I QRNG sono uno strumento relativamente emergente al quale va data la giusta considerazione dati i valori aggiunti che le sue proprietà uniche possono apportare al mondo della crittografia. La proprietà fondamentale che caratterizza i QRNG è la capacità di produrre numeri puramente random e, in alcune varianti, di poter certificare l'affidabilità dei valori prodotti. Tali caratteristiche rendono la considerazione dei QRNG particolarmente interessante per quelle applicazioni dove vi è la necessità di proteggere asset di alto valore o accedere a sistemi critici.
Sebbene quindi per il momento i QRNG sviluppati non possano sostituire i RNG in ogni settore a causa della maggiore complessità e delle prestazioni solitamente inferiori, non va dimenticato che la tecnologia quantica è in continuo sviluppo e in un futuro non troppo lontano nuove implementazioni potrebbero trovare soluzione ai problemi descritti in questa trattazione.