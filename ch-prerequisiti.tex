\chapter{Prerequisiti}

Nel seguente capitolo verranno illustrati i concetti e le proprietà chiave della meccanica quantistica e della casualità, necessari per comprendere al meglio il funzionamento dei generatori quantici di numeri casuali.

\section{Superposition}
Nella meccanica classica un oggetto fisico può avere un solo preciso valore per ogni sua proprietà fisica. Per esempio, un oggetto può essere in un posto nello spazio o in un altro, non in entrambi allo stesso momento. In termini di teoria dell'informazione, questo implica che un bit può solo possedere il valore 0 o il valore 1 in un dato istante, non entrambi. \\Nella meccanica quantistica, invece, tale proprietà non risulta essere valida. Lo stato di un sistema fisico è descritto da un vettore e qualsiasi \textit{superposition}, o più tecnicamente, qualsiasi combinazione lineare di due vettori o stati, descrive uno stato fisico perfettamente valido. 

\section{Casualità della misurazione}
Effettuando una misurazione di una proprietà di un sistema quantico, il risultato sarà un singolo valore appartenente ad un insieme di valori discreti. Per esempio, in meccanica quantistica, è possibile misurare la polarizzazione di fotoni rispetto a diverse direzioni, utilizzando un filtro che permette al fotone di attraversarlo solo se questo è polarizzato nella stessa direzione del filtro. In particolare, per esempio, un filtro orizzontale permetterebbe solo ai fotoni polarizzati orizzontalmente (fotoni nello stato $|H\rangle$) di passare, mentre bloccherebbe quelli polarizzati verticalmente (fotoni nello stato $|V\rangle$). Preparando un fotone nello stato $(|H\rangle + |V\rangle)/\sqrt{2}$ e misurando il suo stato con un filtro, esso avrà esattamente il il 50\% di possibilità di passare. Tale probabilità dipende dai coefficienti della combinazione lineare degli stati H e V. In questo caso la casualità della misurazione è genuina, nel senso che non esiste un modo di predire il risultato della misurazione prima di averla effettuata. Questo indeterminismo è il concetto chiave della meccanica quantistica e quello che permette, come vedremo, la costruzione di generatori di numeri realmente casuali.

\section{Entanglement}
Si definisce entanglement il fenomeno attraverso il quale si viene a creare una speciale connessione tra due elementi quantici. Due oggetti legati da questo rapporto presenteranno sempre una forte correlazione tra loro, anche a grandi distanze spaziali. Per esempio, misurando una proprietà di due fotoni "entangled", questa avrà lo stesso valore per entrambi. Gli scienziati fanno ancora fatica a spiegare questo strano fenomeno, ma numerosi esperimenti continuano a confermare l'esistenza di tale funzionamento nel mondo quantico.

\section{Bell inequality}
Il teorema di Bell incapsula diversi risultati ottenuti in fisica i quali portano alla conclusione che la meccanica quantistica non è compatibile con la teoria delle variabili locali nascoste valida per il mondo classico. Il termine "locale" si riferisce al principio di località, secondo cui una particella può essere influenzata solamente da elementi ad essa vicini e l'interazione mediante campi fisici può avvenire solamente a velocità inferiori a quella della luce. Lo stesso Bell, studiando il concetto di Entanglement osservato nel mondo quantico, ha dimostrato come le variabili di particelle di questo tipo devono essere "non locali" e che quindi esse violano il vincolo definito come Bell inquality secondo cui deve esistere una correlazione matematica tra gli output delle misurazioni di due particelle correlate quantisticamente.

\section{Casualità}
Al fine di comprendere il funzionamento e l'utilità di generatori di numeri casuali è prima necessario definire cosa sia esattamente una fonte di casualità. Per tale definizione saranno considerate sequenze di numeri, in particolare sequenze di 0 e 1. La casualità è strettamente legata all'impossibilità di predire i valori generati. Un semplice test per verificare se una sequenza di numeri è realmente casuale è quello di comprimere il valore con la compressione zip. Se la sequenza compressa risulta essere di dimensione inferiore rispetto a quella di partenza, allora lo strumento ha trovato un pattern ricorrente all'interno della stessa e si può quindi affermare che i valori generati non siano causali. Al fine di ottenere una fonte di valori realmente random essi dovrebbero quindi essere indipendenti gli uni dagli altri. Il problema con questa valutazione è che, considerando una stringa di lunghezza 5, le sequenze 00000, 11111 e 10011 sono tutte equamente probabili. Risulta perciò complicato essere certi della casualità di una serie di valori. I test attualmente utilizzati per misurare la casualità misurano diversi attributi delle sequenze generate, come la presenza di pattern ricorrenti o la distribuzione dei valori all'interno della serie di numeri, che dovrebbe essere uniforme.